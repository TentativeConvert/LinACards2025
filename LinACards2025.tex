\documentclass[10pt]{article}
\usepackage[utf8]{inputenc}
\usepackage[T1]{fontenc}
\usepackage{lmodern}

\usepackage{latex2anki}

% blackboard bold font letters
\newcommand{\CC}{\mathbb{C}}
\newcommand{\FF}{\mathbb{F}}
\newcommand{\NN}{\mathbb{N}}
\newcommand{\QQ}{\mathbb{Q}}
\newcommand{\RR}{\mathbb{R}}
\newcommand{\ZZ}{\mathbb{Z}}
%
% (redefined) symobls 
%
\newcommand{\abs}[1]{\left\vert#1\right\vert}
\newcommand{\norm}[1]{\left\|#1\right\|}
\newcommand{\scprod}[2]{\langle #1, #2\rangle} %Skalarprodukt
\def\phi{\varphi}
\def\epsilon{\varepsilon}
\def\leq{\leqslant}
\def\geq{\geqslant}
\newcommand{\tsmash}{\wedge}
\newcommand{\twedge}{\vee}
\newcommand{\from}{\leftarrow}
\newcommand{\noloc}{:\!}
%
% special maps
%
\newcommand{\id}{\mathrm{id}}
%
% linear algebra
%
\def\vec #1{\mathbf{#1}}
\newcommand{\Potenz}{\mathfrak P}
\newcommand{\hull}[1]{\langle #1 \rangle}
\newcommand{\Lraum}{\mathcal L}

% Categories:
\DeclareMathOperator{\Ar}{Ar}
\newcommand{\cat}[1]{\bm{\mathsf{#1}}}
\newcommand{\iTop}{\underline{\smash{\mathsf{Top}}}}  % internal hom in Top
\newcommand{\iAut}{\underline{\smash{\mathsf{Aut}}}}  
\newcommand{\Aut}{\mathrm{Aut}}
\newcommand{\Mor}{\mathrm{Mor}}
\newcommand{\Hom}{\mathrm{Hom}}
\DeclareMathOperator{\signum}{sgn}
\DeclareMathOperator{\rank}{rk}
\DeclareMathOperator{\im}{im}
\DeclareMathOperator{\Abb}{Abb}
\DeclareMathOperator{\Mat}{Mat}

   
\begin{document}

\begin{note}{LinA-I-Mengen-Morgan-I}
\field De Morgansche Regel (I)
\field
  Das Komplement einer Vereinigung ist \cloze{1}der Schnitt der Komplemente.\clend
\field
\field Mengen
\field Satz
\field 01.08
\end{note}
 
\begin{note}{LinA-I-Mengen-Morgan-II}
\field De Morgansche Regel (II)
\field
    Das Komplement eines Schnitts ist \cloze{1}die Vereinigung der Komplemente.\clend
\field  
\field Mengen
\field Satz
\field 01.08
\end{note}
 
\begin{note}{LinA-I-Mengen-Urbild-Element}
\field
\field
  Ein \textbf{Urbild} eines Elements \(b\in B\) unter einer Abbildung \(f\colon A\to B\) ist \cloze{1}ein Element \(a\in A\) mit \(f(a) = b\).\hint Element\clend
\field
\field Mengen
\field Definition
\field 01.17
\end{note}
 
 
\begin{note}{LinA-I-Mengen-Urbild-Teilmenge}
\field
\field Das \textbf{Urbild} einer Teilmenge \(B'\subset B\) unter einer Abbildung \(f\colon A\to B\) ist \cloze{1}die Menge \emph{aller} \(a\in A\) mit \(f(a) \in B'\).\clend
\field
\field Mengen
\field Definition
\field 01.17
\end{note}
 
\begin{note}{LinA-I-Mengen-Faser}
\field
\field Die \textbf{Faser} eines Elements \(b\in B\) bezüglich einer Abbildung \(f\colon A\to B\) ist \cloze{1}die Menge aller Urbilder von \(b\).\clend
\field
\field Mengen
\field Definition
\field 01.17
\end{note}

\begin{note}{LinA-I-Mengen-injektiv}
  \field
  \field
  Eine Abbildung \(f\colon A\to B\) ist \textbf{injektiv}, wenn \cloze{1}jedes \(b\in B\) \emph{höchstens} ein Urbild besitzt.\hint Urbild\clend
  \field
  \field Mengen
  \field Definition
  \field 01.22
\end{note}

\begin{note}{LinA-I-Mengen-surjektiv}
  \field
  \field
  Eine Abbildung \(f\colon A\to B\) ist \textbf{surjektiv}, wenn \cloze{1}jedes \(b\in B\) \emph{mindestens} ein Urbild besitzt.\hint Urbild\clend
  \field
  \field Mengen
  \field Definition
  \field 01.22
\end{note}

\begin{note}{LinA-I-Mengen-bijektiv}
  \field
  \field
  Eine Abbildung \(f\colon A\to B\) ist \textbf{bijektiv}, wenn \cloze{1}jedes \(b\in B\) \emph{genau} ein Urbild besitzt.\hint Urbild\clend
  \field
  \field Mengen
  \field Definition
  \field 01.22
\end{note}

\begin{note}{LinA-I-Mengen-Umkehrabbildung}
  \field Umkehrabbildung
  \field
  Eine \textbf{Umkehrabbildung} einer Abbildung \(f\colon A\to B\) ist eine \cloze{1}Abbildung \(A\from B \noloc g\) in umgekehrter Richtung, sodass beide Kompositionen \(f\circ g \) und \(g\circ f\) jeweils die Identität sind.\clend
  \field
  \field Mengen
  \field Definition    
  \field 01.20
\end{note}

\begin{note}{LinA-I-Mengen-Iso-von-Mengen}
  \field
  \field
  Ein \textbf{Isomorphismus von Mengen} ist eine \cloze{1}Abbildung von Mengen, die eine Umkehrabbildung besitzt (oder:  die bijektiv ist).\clend
  \field
  \field Mengen
  \field Definition    
  \field 01.20
\end{note}

\begin{note}{LinA-I-Mengen-Umkehr-falls-bijektiv}
  \field
  \field
  Eine Abbildung von Mengen besitzt genau dann eine Umkehrabbildung, wenn sie \cloze{1}bijektiv\clend\ ist.
  \field
  \field Mengen
  \field Satz
  \field 01.23
\end{note}

\begin{note}{LinA-I-Mengen-Mengen-isomorph}
  \field
  \field
  Zwei Mengen sind \textbf{isomorph}, wenn \cloze{1}es eine Bijektion/einen Isomorphismus zwischen ihnen gibt.\clend
  \field
  (Äquivalente Ausdrucksweisen:\par Die Mengen haben dieselbe Kardinalität.\par Die Mengen haben dieselbe Mächtigkeit.  Die Mengen sind gleich mächtig.)
  \field Mengen
  \field Definition
  \field 01.24
\end{note}

\begin{note}{LinA-I-Mengen-Mengen-derselben-Kardinalität}
  \field
  \field
  Zwei Mengen \textbf{haben dieselbe Kardinalität}, wenn \cloze{1}es eine Bijektion/einen Isomorphismus zwischen ihnen gibt.\clend
  \field
  (Äquivalente Ausdrucksweisen:\par Die Mengen sind isomorph.  Die Mengen sind gleich mächtig.  Die Mengen haben dieselbe Mächtigkeit.)
  \field Mengen
  \field Definition
  \field 01.24
\end{note}

\begin{note}{LinA-I-Mengen-Satz-endliche-Mengen-derselben-Kardinalität}
  \field
  \field
  Sind $X$ und $Y$ \cloze{1}endliche\clend\ Mengen mit \cloze{2}gleich vielen Elementen\clend, so sind die folgenden Eigenschaften einer Abbildung \(f\colon X\to Y\) äquivalent zueinander: 
  \begin{center}
    \begin{enumerate}
    \item \(f\) ist injektiv
    \item \(f\) ist surjektiv
    \item \(f\) ist bijektiv
    \end{enumerate}
  \end{center}
  \field
  \field Mengen    
  \field Satz
  \field 01.26
\end{note}

\begin{note}{LinA-I-Mengen-Relation-Teilmenge}
\field
\field
    Eine \textbf{Relation} auf einer Menge \(M\) ist eine Teilmenge von \cloze{1}\(M\times M\).\clend
\field
\field Mengen
\field Definition
\field 01.27
\end{note}

\begin{note}{LinA-I-Mengen-Relation-Schreibweise}
  \field
  \field
  Sei \(R_\sim\subset M\times M\) eine Relation auf \(M\).
  \begin{center}
    \begin{tabular}{ccc}
      \(x\sim y\) & ~bedeutet~ & \cloze{1}\((x,y)\in R_\sim\)\clend
    \end{tabular}
  \end{center}
  \field
  \field Definition
  \field 01.27
\end{note}


\begin{note}{LinA-I-Mengen-Relation-reflexiv}
  \field
  \field
  Eine Relation \(R_\sim\) auf einer Menge \(M\) ist \textbf{reflexiv}, falls
  \cloze{1}
  \[
    x\sim x \text{ für alle } x \in M.
  \]
  \clend
  \field
  \field Mengen
  \field Definition
  \field 01.28
\end{note}

\begin{note}{LinA-I-Mengen-Relation-symmetrisch}
  \field
  \field
  Eine Relation \(R_\sim\)  auf einer Menge \(M\) ist \textbf{symmetrisch}, falls
  \cloze{1}
  \[
    x\sim y  \quad\Leftrightarrow\quad y\sim x
  \]
  für alle \(x,y\in M\).
  \clend
  \field
  \field Mengen
  \field Definition
  \field 01.28
\end{note}

\begin{note}{LinA-I-Mengen-Relation-transitiv}
  \field
  \field
  Eine Relation \(R_\sim\)  auf einer Menge \(M\) ist \textbf{transitiv}, falls
  \cloze{1}
  \[
    (x\sim y \text{ und } y \sim z) \quad\Rightarrow\quad x\sim z
  \]
  für alle \(x,y\in M\).
  \clend
  \field
  \field Mengen
  \field Definition
  \field 01.28
\end{note}

\begin{note}{LinA-I-Mengen-Äquivalenzrelation}
  \field
  \field
  Eine \textbf{Äquivalenzrelation} ist eine 
  \begin{center}
    \cloze{1}\emph{reflexive}\clend\\
    \cloze{2}\emph{symmetrische}\clend\\
    \cloze{3}\emph{transitive}\clend
  \end{center}
  Relation.
  \field
  \field Mengen
  \field Definition
  \field 01.28
\end{note}

\begin{note}{LinA-I-Mengen-Äquivalenzrelation-einer-Abbildung}
  \field
  \field
  Die zu einer Abbildung \(f\colon M \to N\) gehörige Äquivalenzrelation auf \(M\) ist gegeben durch
  \begin{center}
    \begin{tabular}{ccc}
      \(x\sim_f y\) & ~\(\Leftrightarrow\)~ & \cloze{1}\(f(x) = f(y)\).\clend
    \end{tabular}
  \end{center}
  \field
  \field Mengen
  \field Definition
  \field 01.26
\end{note}


\begin{note}{LinA-I-Mengen-Äquivalenzklasse}
  \field
  \field
  Die \textbf{Äquivalenzklasse} \([x]\) eines Elements \(x \in M\) bezüglich einer Relation \(R_\sim\) besteht aus \cloze{1}allen Elementen \(y\) mit \(y\sim x\).\clend
  \field
  \field Mengen
  \field Definition
  \field 01.28
\end{note}

\begin{note}{LinA-I-Mengen-Quotientenmenge}
  \field
  \field
  Die \textbf{Quotientenmenge}
  \[M/\!\sim\]
  einer Menge \(M\) bezüglich einer Relation \(R_\sim\) besteht aus \cloze{1}allen Äquivalenzklassen von \(M\) bezüglich \(R_\sim\).\clend
  \field
  sprich: „\(M\) modulo \(\sim\)“
  \field Mengen
  \field Definition
  \field 01.29
\end{note}


\begin{note}{LinA-I-Mengen-Isomorphiesatz}
\field Isomorphiesatz für Mengen
\field
    Jede Abbildung \(f\colon A\longrightarrow B\) induziert einen Isomorphismus von Mengen
    \cloze{1}
    \[
      A/\!\sim_f \;\xrightarrow{\quad\cong\quad}\; f(A).
    \]
    \clend
    \field
    \field Satz
\field 01.32
\end{note}

\end{document}
