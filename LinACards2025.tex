\documentclass[10pt]{article}
\usepackage[utf8]{inputenc}
\usepackage[T1]{fontenc}
\usepackage{lmodern}

\usepackage{latex2anki}

% blackboard bold font letters
\newcommand{\CC}{\mathbb{C}}
\newcommand{\FF}{\mathbb{F}}
\newcommand{\NN}{\mathbb{N}}
\newcommand{\QQ}{\mathbb{Q}}
\newcommand{\RR}{\mathbb{R}}
\newcommand{\ZZ}{\mathbb{Z}}
%
% (redefined) symobls 
%
\newcommand{\abs}[1]{\left\vert#1\right\vert}
\newcommand{\norm}[1]{\left\|#1\right\|}
\newcommand{\scprod}[2]{\langle #1, #2\rangle} %Skalarprodukt
\def\phi{\varphi}
\def\epsilon{\varepsilon}
\def\leq{\leqslant}
\def\geq{\geqslant}
\newcommand{\tsmash}{\wedge}
\newcommand{\twedge}{\vee}
\newcommand{\from}{\leftarrow}
\newcommand{\noloc}{:\!}
%
% special maps
%
\newcommand{\id}{\mathrm{id}}
%
% linear algebra
%
\def\vec #1{\mathbf{#1}}
\newcommand{\Potenz}{\mathfrak P}
\newcommand{\hull}[1]{\langle #1 \rangle}
\newcommand{\Lraum}{\mathcal L}

% Categories:
\DeclareMathOperator{\Ar}{Ar}
\newcommand{\cat}[1]{\bm{\mathsf{#1}}}
\newcommand{\iTop}{\underline{\smash{\mathsf{Top}}}}  % internal hom in Top
\newcommand{\iAut}{\underline{\smash{\mathsf{Aut}}}}  
\newcommand{\Aut}{\mathrm{Aut}}
\newcommand{\Mor}{\mathrm{Mor}}
\newcommand{\Hom}{\mathrm{Hom}}
\DeclareMathOperator{\signum}{sgn}
\DeclareMathOperator{\rank}{rk}
\DeclareMathOperator{\im}{im}
\DeclareMathOperator{\Abb}{Abb}
\DeclareMathOperator{\Mat}{Mat}

   
\begin{document}

\begin{note}{LinA-I-Mengen-Morgan-I}
\field De Morgansche Regel (I)
\field
  Das Komplement einer Vereinigung ist \cloze{1}der Schnitt der Komplemente.\clend
\field
\field Mengen
\field Satz
\field 01.08
\end{note}
 
\begin{note}{LinA-I-Mengen-Morgan-II}
\field De Morgansche Regel (II)
\field
    Das Komplement eines Schnitts ist \cloze{1}die Vereinigung der Komplemente.\clend
\field  
\field Mengen
\field Satz
\field 01.08
\end{note}
 
\begin{note}{LinA-I-Mengen-Urbild-Element}
\field
\field
  Ein \textbf{Urbild} eines Elements \(b\in B\) unter einer Abbildung \(f\colon A\to B\) ist \cloze{1}ein Element \(a\in A\) mit \(f(a) = b\).\hint Element\clend
\field
\field Mengen
\field Definition
\field 01.17
\end{note}
 
 
\begin{note}{LinA-I-Mengen-Urbild-Teilmenge}
\field
\field Das \textbf{Urbild} einer Teilmenge \(B'\subset B\) unter einer Abbildung \(f\colon A\to B\) ist \cloze{1}die Menge \emph{aller} \(a\in A\) mit \(f(a) \in B'\).\clend
\field
\field Mengen
\field Definition
\field 01.17
\end{note}
 
\begin{note}{LinA-I-Mengen-Faser}
\field
\field Die \textbf{Faser} eines Elements \(b\in B\) bezüglich einer Abbildung \(f\colon A\to B\) ist \cloze{1}die Menge aller Urbilder von \(b\).\clend
\field
\field Mengen
\field Definition
\field 01.17
\end{note}

\begin{note}{LinA-I-Mengen-injektiv}
  \field
  \field
  Eine Abbildung \(f\colon A\to B\) ist \textbf{injektiv}, wenn \cloze{1}jedes \(b\in B\) \emph{höchstens} ein Urbild besitzt.\hint Urbild\clend
  \field
  \field Mengen
  \field Definition
  \field 01.22
\end{note}

\begin{note}{LinA-I-Mengen-surjektiv}
  \field
  \field
  Eine Abbildung \(f\colon A\to B\) ist \textbf{surjektiv}, wenn \cloze{1}jedes \(b\in B\) \emph{mindestens} ein Urbild besitzt.\hint Urbild\clend
  \field
  \field Mengen
  \field Definition
  \field 01.22
\end{note}

\begin{note}{LinA-I-Mengen-bijektiv}
  \field
  \field
  Eine Abbildung \(f\colon A\to B\) ist \textbf{bijektiv}, wenn \cloze{1}jedes \(b\in B\) \emph{genau} ein Urbild besitzt.\hint Urbild\clend
  \field
  \field Mengen
  \field Definition
  \field 01.22
\end{note}

\begin{note}{LinA-I-Mengen-Umkehrabbildung}
  \field Umkehrabbildung
  \field
  Eine \textbf{Umkehrabbildung} einer Abbildung \(f\colon A\to B\) ist eine \cloze{1}Abbildung \(A\from B \noloc g\) in umgekehrter Richtung, sodass beide Kompositionen \(f\circ g \) und \(g\circ f\) jeweils die Identität sind.\clend
  \field
  \field Mengen
  \field Definition    
  \field 01.20
\end{note}

\begin{note}{LinA-I-Mengen-Iso-von-Mengen}
  \field
  \field
  Ein \textbf{Isomorphismus von Mengen} ist eine \cloze{1}Abbildung von Mengen, die eine Umkehrabbildung besitzt (oder:  die bijektiv ist).\clend
  \field
  \field Mengen
  \field Definition    
  \field 01.20
\end{note}

\begin{note}{LinA-I-Mengen-Umkehr-falls-bijektiv}
  \field
  \field
  Eine Abbildung von Mengen besitzt genau dann eine Umkehrabbildung, wenn sie \cloze{1}bijektiv\clend\ ist.
  \field
  \field Mengen
  \field Satz
  \field 01.23
\end{note}

\begin{note}{LinA-I-Mengen-Mengen-isomorph}
  \field
  \field
  Zwei Mengen sind \textbf{isomorph}, wenn \cloze{1}es eine Bijektion/einen Isomorphismus zwischen ihnen gibt.\clend
  \field
  (Äquivalente Ausdrucksweisen:\par Die Mengen haben dieselbe Kardinalität.\par Die Mengen haben dieselbe Mächtigkeit.  Die Mengen sind gleich mächtig.)
  \field Mengen
  \field Definition
  \field 01.24
\end{note}

\begin{note}{LinA-I-Mengen-Mengen-derselben-Kardinalität}
  \field
  \field
  Zwei Mengen \textbf{haben dieselbe Kardinalität}, wenn \cloze{1}es eine Bijektion/einen Isomorphismus zwischen ihnen gibt.\clend
  \field
  (Äquivalente Ausdrucksweisen:\par Die Mengen sind isomorph.  Die Mengen sind gleich mächtig.  Die Mengen haben dieselbe Mächtigkeit.)
  \field Mengen
  \field Definition
  \field 01.24
\end{note}

\begin{note}{LinA-I-Mengen-Satz-endliche-Mengen-derselben-Kardinalität}
  \field
  \field
  Sind $X$ und $Y$ \cloze{1}endliche\clend\ Mengen mit \cloze{2}gleich vielen Elementen\clend, so sind die folgenden Eigenschaften einer Abbildung \(f\colon X\to Y\) äquivalent zueinander: 
  \begin{center}
    \begin{enumerate}
    \item \(f\) ist injektiv
    \item \(f\) ist surjektiv
    \item \(f\) ist bijektiv
    \end{enumerate}
  \end{center}
  \field
  \field Mengen    
  \field Satz
  \field 01.26
\end{note}

\begin{note}{LinA-I-Mengen-Relation-Teilmenge}
\field
\field
    Eine \textbf{Relation} auf einer Menge \(M\) ist eine Teilmenge von \cloze{1}\(M\times M\).\clend
\field
\field Mengen
\field Definition
\field 01.27
\end{note}

\begin{note}{LinA-I-Mengen-Relation-Schreibweise}
  \field
  \field
  Sei \(R_\sim\subset M\times M\) eine Relation auf \(M\).
  \begin{center}
    \begin{tabular}{ccc}
      \(x\sim y\) & ~bedeutet~ & \cloze{1}\((x,y)\in R_\sim\)\clend
    \end{tabular}
  \end{center}
  \field
  \field Definition
  \field 01.27
\end{note}


\begin{note}{LinA-I-Mengen-Relation-reflexiv}
  \field
  \field
  Eine Relation \(R_\sim\) auf einer Menge \(M\) ist \textbf{reflexiv}, falls
  \cloze{1}
  \[
    x\sim x \text{ für alle } x \in M.
  \]
  \clend
  \field
  \field Mengen
  \field Definition
  \field 01.28
\end{note}

\begin{note}{LinA-I-Mengen-Relation-symmetrisch}
  \field
  \field
  Eine Relation \(R_\sim\)  auf einer Menge \(M\) ist \textbf{symmetrisch}, falls
  \cloze{1}
  \[
    x\sim y  \quad\Leftrightarrow\quad y\sim x
  \]
  für alle \(x,y\in M\).
  \clend
  \field
  \field Mengen
  \field Definition
  \field 01.28
\end{note}

\begin{note}{LinA-I-Mengen-Relation-transitiv}
  \field
  \field
  Eine Relation \(R_\sim\)  auf einer Menge \(M\) ist \textbf{transitiv}, falls
  \cloze{1}
  \[
    (x\sim y \text{ und } y \sim z) \quad\Rightarrow\quad x\sim z
  \]
  für alle \(x,y\in M\).
  \clend
  \field
  \field Mengen
  \field Definition
  \field 01.28
\end{note}

\begin{note}{LinA-I-Mengen-Äquivalenzrelation}
  \field
  \field
  Eine \textbf{Äquivalenzrelation} ist eine 
  \begin{center}
    \cloze{1}\emph{reflexive}\clend\\
    \cloze{2}\emph{symmetrische}\clend\\
    \cloze{3}\emph{transitive}\clend
  \end{center}
  Relation.
  \field
  \field Mengen
  \field Definition
  \field 01.28
\end{note}

\begin{note}{LinA-I-Mengen-Äquivalenzrelation-einer-Abbildung}
  \field
  \field
  Die zu einer Abbildung \(f\colon M \to N\) gehörige Äquivalenzrelation auf \(M\) ist gegeben durch
  \begin{center}
    \begin{tabular}{ccc}
      \(x\sim_f y\) & ~\(\Leftrightarrow\)~ & \cloze{1}\(f(x) = f(y)\).\clend
    \end{tabular}
  \end{center}
  \field
  \field Mengen
  \field Definition
  \field 01.26
\end{note}


\begin{note}{LinA-I-Mengen-Äquivalenzklasse}
  \field
  \field
  Die \textbf{Äquivalenzklasse} \([x]\) eines Elements \(x \in M\) bezüglich einer Relation \(R_\sim\) besteht aus \cloze{1}allen Elementen \(y\) mit \(y\sim x\).\clend
  \field
  \field Mengen
  \field Definition
  \field 01.28
\end{note}

\begin{note}{LinA-I-Mengen-Quotientenmenge}
  \field
  \field
  Die \textbf{Quotientenmenge}
  \[M/\!\sim\]
  einer Menge \(M\) bezüglich einer Relation \(R_\sim\) besteht aus \cloze{1}allen Äquivalenzklassen von \(M\) bezüglich \(R_\sim\).\clend
  \field
  sprich: „\(M\) modulo \(\sim\)“
  \field Mengen
  \field Definition
  \field 01.29
\end{note}


\begin{note}{LinA-I-Mengen-Isomorphiesatz}
\field Isomorphiesatz für Mengen
\field
    Jede Abbildung \(f\colon A\longrightarrow B\) induziert einen Isomorphismus von Mengen
    \cloze{1}
    \[
      A/\!\sim_f \;\xrightarrow{\quad\cong\quad}\; f(A).
    \]
    \clend
    \field
    \field Satz
\field 01.32
\end{note}


\begin{note}{LinA-I-Gruppen-Verknüpfung}
  \field
  \field
  Eine (binäre) \textbf{Verknüpfung} auf einer Menge \(M\) ist \cloze{1}eine Abbildung \(M\times M\to M\).\clend
  \field
  \field Gruppen
  \field Definition
  \field 02.01
\end{note}

\begin{note}{LinA-I-Gruppen-Verknüpfung-assoziativ}
  \field
  \field
  Eine Verknüpfung \(\star\) auf einer Menge \(M\) ist \textbf{assoziativ}, falls gilt:
  \cloze{1}
  \[
    x\star(y\star z) = (x\star y)\star z
  \]
  für alle \(x,y,z \in M\).
  \clend
  \field
  \field Definition
  \field 02.02
\end{note}

\begin{note}{LinA-I-Gruppen-Verknüpfung-kommutativ}
  \field
  \field
  Eine Verknüpfung \(\star\) auf einer Menge \(M\) ist \textbf{kommutativ}, falls gilt:
  \cloze{1}
  \[
    x\star y = y\star x
  \]
  für alle \(x,y \in M\).
  \clend
  \field
  \field Gruppen
  \field Definition
  \field 02.02
\end{note}

\begin{note}{LinA-I-Gruppen-Verknüpfung-neutrales-Element}
  \field
  \field
    Ein \textbf{neutrales Element} einer Verknüpfung \(\star\) auf \(M\) ist ein Element \(e\in M\), für das gilt:
    \cloze{1}
    \[
      e\star x = x = x\star e
    \]
    für alle \(x\in M\).
    \clend
\field Definition
\field 02.03
\end{note}

\begin{note}{LinA-I-Gruppen-Verknüpfung-inverses-Element}
  \field
  \field
  Sei \(\star\) eine Verknüpfung auf \(M\) mit neutralem Element~\(e\). 
  Ein \textbf{inverses Element} zu \(x\in M\) ist ein \(y\in M\), für das gilt:
  \cloze{1}
  \[
    x\star y = e = y\star x
  \]
  \clend
  \field
  \field Gruppen
  \field Definition  
  \field 02.03
\end{note}

\begin{note}{LinA-I-Gruppen-Def-Gruppe-a}
  \field
  \field
  Eine \textbf{Gruppe} besteht aus einer Menge \(G\) und aus \cloze{1}einer Verknüpfung
  \(\star\) auf \(G\).\clend
  \field
  \field Gruppen
  \field Definition
  \field 02.03
\end{note}

\begin{note}{LinA-I-Gruppen-Def-Gruppe-b}
  \field
  \field
  Eine \textbf{Gruppe} \((G,\star)\) muss den folgenden Axiomen genügen:
  \begin{center}
    \begin{itemize}
    \item[(G1)] \cloze{1}\(\star\) ist assoziativ.\clend
    \item[(G2)] \cloze{2}\(\star\) besitzt ein neutrales Element.\clend
    \item[(G3)] \cloze{3}Jedes \(g\in G\) besitzt ein Inverses bezüglich \(\star\).\clend
    \end{itemize}
  \end{center}
  \field
  \field Gruppen
  \field Definition
  \field 02.03
\end{note}

\begin{note}{LinA-I-Gruppen-abelsch}
  \field
  \field
  Ein Gruppe \((G,\star)\) ist \textbf{abelsch}, falls \cloze{1}die Verknüpfung \(\star\) kommutativ ist.\clend
  \field
  In diesem Fall benutzen wir in der Regel das Symbol „\(+\)“ für die Verknüpfung.
  \field Gruppen
  \field Definition
  \field 02.03
\end{note}

\begin{note}{LinA-I-Gruppen-Inverses-des-Produkts}
  \field
  \field
  In einer Gruppe \((G,\cdot)\) ist das Inverse eines Produkts \(x\cdot y\) gegeben durch
  \begin{tabular}{rcl}
    \((x\cdot y)^{-1}\) & \(~=~\) & \cloze{1}\(y^{-1}\cdot x^{-1}\)\clend
  \end{tabular}
  \field
  \emph{Reihenfolge beachten!}
  \field Gruppen
  \field Satz    
  \field 02.07
\end{note}

\begin{note}{LinA-I-Gruppen-Def-Untergruppe}
  \field
  \field
  Eine \textbf{Untergruppe} einer Gruppe \((G,\cdot)\) ist eine Teilmenge \(H\subset G\), für die gilt:
  \begin{enumerate}
  \item Die Verknüpfung \(\cdot\) lässt sich \cloze{1}einschränken zu einer Verknüpfung \(H\times H\to H\).\clend
  \item \cloze{2}Zusammen mit der eingeschränkten Verknüpfung ist \(H\) selbst wieder eine Gruppe.\clend
  \end{enumerate}
  \field
  \field Gruppen
  \field Definition
  \field 02.08
\end{note}

\begin{note}{LinA-I-Gruppen-Kriterien-Untergruppe}
  \field
  \field
  Sei \((G,\cdot)\) eine Gruppe mit neutralem Element \(1_G\). Eine Teilmenge \(H\subset G\) ist genau dann eine Untergruppe, wenn gilt:
  \begin{center}
    \begin{enumerate}[(i)]
    \item \cloze{1}\(1_G \in H\)\clend
    \item \cloze{2}\(\forall x,y\in H\colon x\cdot y \in H\)\clend
    \item \cloze{3}\(\forall x\in H\colon \quad x^{-1} \in H\)\clend
    \end{enumerate}
  \end{center}
  \field
  \field Gruppen
  \field Notiz    
  \field 02.09
\end{note}

\begin{note}{LinA-I-Gruppen-Homomorphismus}
  \field
  \field
  Eine Abbildung zwischen Gruppen \(f\colon (G,\star)\to(H,\ast)\) ist ein \textbf{Gruppenhomomorphismus}, falls gilt:
  \cloze{1}
  \[
    f(g_1\star g_2) = f(g_1)\ast f(g_2)
  \]
  für alle \(g_1,g_2 \in G\).
  \clend
  \field
  \field Gruppen
  \field Definition   
  \field 02.10
\end{note}

\begin{note}{LinA-I-Gruppen-Homomorphismus-Kern}
  \field
  \field
  Der \textbf{Kern} eines Gruppenhomomorphismus \[f\colon G_1\to G_2\] besteht aus \cloze{1}allen Elementen von \(G_1\), die auf das neutrale Element von \(G_2\) abgebildet werden.\clend
  \field
  \field Gruppen    
  \field Definition
  \field 02.15
\end{note}

\begin{note}{LinA-I-Gruppen-Untergruppe-Linksnebenklasse}
  \field
  \field
  Die \textbf{Linksnebenklasse} eines Gruppenelements \(g\) bezüglich einer Untergruppe \(H\) von \((G,\star)\) ist die Teilmenge
  \cloze{1}
  \[
    g\star H := \{g\star h \mid h\in H\}.
  \]
  \clend
  \field
  \field Gruppen
  \field Definition    
  \field 02.17
\end{note}

\begin{note}{LinA-I-Gruppen-Untergruppe-normal}
  \field
  \field
  Eine Untergruppe ist \textbf{normal}, wenn \cloze{1}Links- und Rechtsnebenklassen übereinstimmen.\clend
  \field
  (\(g\star H = H\star g\) für alle \(g\in G\).)
  \field Gruppen
  \field Definition    
  \field 02.19
\end{note}

\begin{note}{LinA-I-Gruppen-Untergruppe-Linksnebenklassen-Gruppenstruktur}
  \field
  \field
  Die Menge aller Linksnebenklassen einer Untergruppe besitzt eine natürliche Gruppenstruktur, sofern \cloze{1}die Untergruppe normal ist.\clend
  \field
  (In diesem Fall sind Linksnebenklassen gleich Rechtsnebenklassen, und die Menge aller dieser Nebenklassen mit obiger Gruppenstruktur heißt Quotientengruppe.)
  \field Gruppen
  \field Satz  
  \field 2.21
\end{note}

\begin{note}{LinA-I-Gruppen-Quotientengruppe}
  \field
  \field
  Die \textbf{Quotientengruppe} \(G/H\) einer Gruppe modulo einer normalen Untergruppe besteht aus \cloze{1}den (Rechts- = Links-)\allowbreak Nebenklassen von \(H\) in \(G\).\clend
  \field
  \field Gruppen
  \field Definition
  \field 02.21
\end{note}

\begin{note}{LinA-I-Gruppen-Def-Permutation}
  \field
  \field
  Eine \textbf{\(n\)-stellige Permutation} ist \cloze{1}eine bijektive Abbildung
  \[
    \{1,\dots, n\} \xrightarrow{\quad\cong\quad} \{1,\dots,n\}.
  \]
  \clend
  \field
  \field Gruppen
  \field Definition  
  \field 02.26
\end{note}

\begin{note}{LinA-I-Gruppen-Def-symmetrische-Gruppe-a}
  \field
  \field
  Die \textbf{symmetrische Gruppe} \(S_n\) besteht aus \cloze{1}allen \(n\)-stelligen Permutationen.\clend
  \field
  \field Gruppen
  \field Definition
  \field 02.27
\end{note}

\begin{note}{LinA-I-Gruppen-Def-symmetrische-Gruppe-b}
  \field
  \field
  Die \textbf{symmetrische Gruppe} \(S_n\) ist die Menge aller \(n\)-stelliger Permutationen zusammen mit \cloze{2}der Komposition\clend\ als Verknüpfung.
  \field
  \field Gruppen
  \field Definition  
  \field 02.27
\end{note}

\begin{note}{LinA-I-Gruppen-Def-Permutation-Fehlstand}
  \field
  \field
  Ein \textbf{Fehlstand} einer Permutation \(\sigma\in S_n\) ist \cloze{1}eine zwei-elementige Teilmenge \(\{i,k\}\subseteq\{1,\dots, n\}\) mit
  \begin{align*}
    && i&<k &&\text{ und } &\sigma(i) &> \sigma(k)
  \end{align*}
  \clend
  \field
  \field Gruppen
  \field Definition 
  \field 02.29
\end{note}

\begin{note}{LinA-I-Gruppen-Def-Permutation-Signum}
  \field
  \field
  Das \textbf{Signum} einer Permutation \(\sigma\) ist gegeben durch
  \begin{tabular}{rcl}
    \(\signum(\sigma)\) & \(~:=~\) & \cloze{1}\((-1)^{\text{Anzahl der Fehlstände von \(\sigma\)}}\)\clend
  \end{tabular}
  \field
  \field Gruppen
  \field Definition
  \field 02.29
\end{note}

\begin{note}{LinA-I-Gruppen-Permutation-Signum-Homomorphismus}
  \field
  \field
  Das Signum definiert einen Gruppenhomomorphismus \(S_n\to \{\pm 1\}\).  Das bedeutet:
  \cloze{1}
  \[
    \signum(\sigma_1\circ\sigma_2) = \signum(\sigma_1)\cdot\signum(\sigma_2)
  \]
  \clend
  \field
  \field Gruppen
  \field Satz   
  \field 02.30
\end{note}

\begin{note}{LinA-I-Gruppen-Def-alternierende-Gruppe}
  \field
  \field
  Die \textbf{alternierende Gruppe} \(A_n\) ist \cloze{1}der Kern des Signums.\clend
  \field
  \[
    A_n := \ker(\signum\colon S_n \to \{\pm 1\})
  \]
  \field Gruppen
  \field Definition
  \field 02.31
\end{note}


\end{document}
